\documentclass{beamer}
\mode<presentation> { 
\usetheme{Frankfurt}
\setbeamertemplate{footline}[frame number]
\usecolortheme{whale}
%\setbeamercolor{frametitle}{fg=darkblue, bg=white} 
%\usecolortheme{rose} 
\setbeamercovered{transparent}
 }
\usepackage[spanish]{babel} 
\usepackage{graphicx,wrapfig}
\usepackage[utf8]{inputenc} 
\usepackage[T1]{fontenc} 
\usepackage{lmodern}
\usepackage[all]{xy}
\usepackage{ragged2e}
\usepackage{amsmath,amssymb,amsfonts,latexsym,cancel}
%\usepackage{wrapfig}


\title{Principio de Hamilton -Dinámica Lagrangiana y Hamiltoniana-} 
%\institute{Benem\'erita Universidad Aut\'onoma de Puebla}
\author[José Augusto de la Fuente León, \\ David Pavel] {\textbf{José Augusto de la Fuente León, \\ \ \\David Pavél Juárez López}\\ \ \\
Profesor:\\
\emph{Dr. Delepine}}
\date{Julio 8, 2013}
%\logo{\includegraphics[height=0.95cm]{escudo.jpg}}

%%%%%%%%%%%%%%%%%%%%%%%%%%%%%%%%%%%%%%%%%%%%%%%%%%%%%%%%%%%%%%%%%%%%%%%%%%%%%%%%%%%%
%\makeatletter
%%Feynman slash
%\newbox\slashbox \setbox\slashbox=\hbox{$/$}
%\newbox\Slashbox \setbox\Slashbox=\hbox{\large$/$}
%\def\pFMslash#1{\setbox\@tempboxa=\hbox{$#1$}
%  \@tempdima=0.5\wd\slashbox \advance\@tempdima 0.5\wd\@tempboxa
%  \copy\slashbox \kern-\@tempdima \box\@tempboxa}
%\def\pFMSlash#1{\setbox\@tempboxa=\hbox{$#1$}
%  \@tempdima=0.5\wd\Slashbox \advance\@tempdima 0.5\wd\@tempboxa
%  \copy\Slashbox \kern-\@tempdima \box\@tempboxa}
%\def\FMslash{\protect\pFMslash}
%\def\FMSlash{\protect\pFMSlash}
%\def\miss#1{\ifmmode{/\mkern-11mu #1}\else{${/\mkern-11mu #1}$}\fi}
%%%%% Uso:  \pFMSlash{p}
%\makeatother
%%%%%%%%%%%%%%%%%%%%%%%%%%%%%%%%%%%%%%%%%%%%%%%%%%%%%%%%%%%%%%%%%%%%%%%%%%%%%%%%%%

\begin{document}
\begin{frame} 
%%\pgfimage[height=2.5cm]{buap.jpg}
%%\begin{picture}(1.5,0.7)
%%\put(2,-60){\includegraphics[width=1.8cm]{buap.jpg}}
%%\end{picture}
\titlepage
\end{frame} 



%%%%%%%%%%%%%%%%%%%%%%%%%%%%%%%%%%%%%%%%%%%%%%%%%%%%%%%%%%%%%%%%%%%%%%%%%%%%%%%%%%%%%%%%%%%%%%%%%%%%%%%%%%%%%%
%%%%%%%%%%%%%%%%%%%%%%%%%%%%%%%%%%%%%%%%%%%%%%%%%%%%%%%%%%%%%%%%%%%%%%%%%%%%%%%%%%%%%%%%%%%%%%%%%%%%%%%%%%%%%%

\section[]{Introducción}
%\subsection{}
\begin{frame}
\frametitle{Introducción}
\justifying
\begin{itemize}
\justifying
\item Experiencias han mostrado que el movimiento de una partícula, en un marco de referencia inercial, es correctamente descrita por la ecuación de Newton $\vec{F} = \dot{\vec{p}}$.
\item Tal método es contenido en el principio de Hamilton  y la ecuación de movimiento resulta de la aplicación del principio llamado ecuaciones de Lagrange
\end{itemize}
 \end{frame}
 
 

%%%%%%%%%%%%%%%%%%%%%%%%%%%%%%%%%%%%%%%%%%%%%%%%%%%%%%%%%%%%%%%%%%%%%%%%%%%%%%%%%%%%%%%%%%%%%%
%%%%%%%%%%%%%%%%%%%%%%%%%%%%%%%%%%%%%%%%%%%%%%%%%%%%%%%%%%%%%%%%%%%%%%%%%%%%%%%%%%%%%%%%%%%%%%

\begin{frame}
\justifying
\frametitle{Principio de Hamilton}
\textit{De todos los posibles caminos a lo largo del cual un sistema dinámico puede moverse de un punto a otro con un intervalo de tiempo especifico (consistente con cualquier restricción), el actual camino seguido es en el cual minimiza la integral temporal de la diferencia entre la energía cinética y la energía potencial}
\end{frame}


\begin{frame}
\justifying
\frametitle{}
En términos de calculo de variaciones, el principio de Hamilton se expresa como,
\begin{equation}
\label{ec1}
\delta \int_{t_1}^{t_2} (T-U) dt =0 
\end{equation}
donde el símbolo $\delta$ es una abreviatura para describir la variación.
\\ Ahora 
\begin{equation}
T=T(\dot{x}_i), \qquad U=U(x_i)
\end{equation}
Si definimos la diferencia de estas cantidades, como 
\begin{equation}
L \equiv T -U =L(x_i, \;\dot{x}_i)
\end{equation}
\end{frame}
%
\begin{frame}
\justifying
\frametitle{}
Entonces la Ec.~\ref{ec1} se transforma como,
\begin{equation}
\label{1}
\delta \int_{t_1}^{t_2} L(x_i, \;\dot{x}_i) dt =0 
\end{equation}
o, efectuando la variación:
\begin{eqnarray}
\int_{t_1}^{t_2}\left( \frac{\partial L}{\partial x_i} - \frac{d}{dt} \frac{\partial L}{\partial \dot{x_i}} \right) \delta x dt =0 
\end{eqnarray}
\end{frame}
%
\begin{frame}
\justifying
\frametitle{}
Consecuentemente se obtiene la ecuación:
\begin{equation}
\frac{\partial L}{\partial x_i} - \frac{d}{dt} \frac{\partial L}{\partial \dot{x_i}} = 0
\end{equation}
Estas son las ecuaciones de movimiento de Lagrange para la partícula, y la cantidad L es llamada la función Lagrangiana o el lagrangiano de la partícula.
\end{frame}
%
\begin{frame}
\justifying
\frametitle{Coordenadas Generalizadas}
\begin{itemize}
\justifying
\item Para determinar la posición de un sistema de $n$ puntos materiales en el espacio, hace falta dar $n$ vectores de posición, es decir, $3n$ coordenadas. \\ \ \\
\item  Si existen $m$ ecuaciones de constricción que relaciona alguna de esta coordenada con otra, entonces $3n-m$ coordenadas son independientes, y se dice que el sistema posee $3n-m=s$ grados de libertad.
\end{itemize}
\end{frame}
%
\begin{frame}
\justifying
\frametitle{}
\begin{itemize}
\justifying
\item Damos el nombre de coordenadas generalizadas a cualquier conjunto de cantidades que defina completamente la posición de un sistema.
\item Las coordenadas generalizadas se acostumbra a escribirlas como $q_1, \; q_2,\ldots,$ o simplemente como $q_j$. Las derivadas $\dot{q}_j$ sus velocidades generalizadas.
\end{itemize}
\end{frame}
%
\begin{frame}
\justifying
\frametitle{Ecuaciones de Movimiento de Lagrange en Coordenadas Generalizadas}
\begin{itemize}
\justifying
\item La lagrangiana para un sistema es definida a ser la diferencia entre la energía cinética y la energía potencial.
\item Pero energía es una cantidad escalar y por lo tanto la Lagrangiana es una función escalar.
\item Por lo que la Lagrangiana es invariante con respecto a transformación de coordenadas.
\end{itemize}
\end{frame}
%
\begin{frame}
\justifying
\frametitle{}
 Expresamos la lagrangiana en terminos de $q_j$ y $\dot{q}_j$: 
\begin{eqnarray}
L &=& T(q_j, \, \dot{q}_j, \,t) - U(q_j,\,t) \nonumber \\
 &=& L(q_j, \, \dot{q}_j, \,t)
\end{eqnarray}
Entonces el principio de Hamilton se expresa,
\begin{equation}
\delta \int_{t_1}^{t_2} L(q_j, \;\dot{q}_j) dt =0 
\end{equation}
\end{frame}

\begin{frame}
\justifying
\frametitle{}
Efectuando la variación:
\begin{eqnarray}
\int_{t_1}^{t_2}\left( \frac{\partial L}{\partial q} - \frac{d}{dt} \frac{\partial L}{\partial \dot{q}} \right) \delta q dt =0 
\end{eqnarray}
y consecuentemente se obtiene la ecuación:
\begin{equation}
\frac{\partial L}{\partial x_i} - \frac{d}{dt} \frac{\partial L}{\partial \dot{x_i}} = 0
\end{equation}
Estas son las ecuaciones de movimiento de Euler-Lagrange del sistema. 
\end{frame}

%\begin{frame}
%\justifying
%\frametitle{}
%
%\end{frame}
%
%\begin{frame}
%\justifying
%\frametitle{}
%
%\end{frame}
%
%\begin{frame}
%\justifying
%\frametitle{}
%
%\end{frame}
%
%\begin{frame}
%\justifying
%\frametitle{}
%
%\end{frame}
%
%\begin{frame}
%\justifying
%\frametitle{}
%
%\end{frame}
%
%\begin{frame}
%\justifying
%\frametitle{}
%
%\end{frame}
%
%\begin{frame}
%\justifying
%\frametitle{}
%
%\end{frame}
%
%\begin{frame}
%\justifying
%\frametitle{}
%
%\end{frame}
%
%\begin{frame}
%\justifying
%\frametitle{}
%
%\end{frame}
%\begin{frame}
%
%\end{frame}
%%%%%%%%%%%%%%%%%%%%%%%%%%%%%%%%%%%%%%%%%%%%%%%%%%%%%%%%%%%%%%%%%%%%%%%%%%%%%%%%%%%%%%%%%%%%%%
%%%%%%%%%%%%%%%%%%%%%%%%%%%%%%%%%%%%%%%%%%%%%%%%%%%%%%%%%%%%%%%%%%%%%%%%%%%%%%%%%%%%%%%%%%%%%%

%\section[]{}
%%\subsection{}
%\begin{frame}
%\justifying
%\frametitle{}
%
%\end{frame}



%%%%%%%%%%%%%%%%%%%%%%%%%%%%%%%%%%%%%%%%%%%%%%%%%%%%%%%%%%%%%%%%%%%%%%%%%%%%%%%%%%%%%%%%%%%%%%
%%%%%%%%%%%%%%%%%%%%%%%%%%%%%%%%%%%%%%%%%%%%%%%%%%%%%%%%%%%%%%%%%%%%%%%%%%%%%%%%%%%%%%%%%%%%%%







%
%\begin{thebibliography}{99}
%
%\bibliography{test}
%
%\section[Bibliografía]{Bibliografía}
%%\subsection{}
%
%\begin{frame}
%\justifying
%\frametitle{Bibliografía}
%
%\bibitem{ME} Ver por ejemplo, M. E. Peskin and D. V. Schroeder,
%\textit{An introduction to quantum field theory}, Addison--Wesley
%P.C. (1996).
%\end{frame}
%\end{thebibliography}


\end{document}